\documentclass[aspectratio=169]{beamer}
\usepackage[style=ieee,backend=biber]{biblatex}
\addbibresource{reference.bib}
\usepackage{colortbl,tabularx,mathrsfs,calligra}
\usepackage{amsmath,amsfonts,amssymb,amsthm}
\usepackage{ragged2e}
\usepackage[bahasa]{babel}
\usepackage{tikz}
\usepackage{caption}
\usepackage{wrapfig}
\usepackage{multirow}
\usepackage{multicol}
\usepackage{array}
\usepackage{pgfplots, tkz-euclide,calc}
\pgfplotsset{compat=1.18}
\usepackage{listings}

\graphicspath{{C:/Users/teoso/OneDrive/Documents/Tugas Kuliah/Template Math Depart/}{./foto/}}

\definecolor{HIMAmuda}{HTML}{01D1FD}
\definecolor{HIMAtua}{HTML}{02016A}
\definecolor{HIMAabu}{HTML}{CBCBCC}
\definecolor{PastelGreen}{HTML}{77DD77}

\usetheme{Madrid}

\setbeamercolor{palette primary}{bg=HIMAtua,fg=white}
\setbeamercolor{palette secondary}{bg=HIMAmuda,fg=black}
\setbeamercolor{palette tertiary}{bg=HIMAabu,fg=black}
\setbeamercolor{palette quaternary}{bg=HIMAmuda,fg=white}
\setbeamercolor{structure}{fg=HIMAmuda} % itemize, enumerate, etc
\setbeamercolor{section in toc}{fg=HIMAtua} % TOC sections
\setbeamercolor{bibliography item}{parent=palette secondary}
\setbeamercolor*{bibliography entry author}{parent=section in toc}

\usetikzlibrary{shapes.geometric, arrows}

\tikzstyle{startstop} = [ellipse, minimum width=1cm, minimum height=1cm,text centered, draw=black, fill=red!30]
\tikzstyle{process} = [rectangle, minimum width=2cm, minimum height=1cm, text centered, draw=black, fill=blue!30]
\tikzstyle{decision} = [diamond, minimum width=1cm, minimum height=1cm, text centered, draw=black, fill=blue!50]
\tikzstyle{arrow} = [thick,->,>=stealth]

\newcolumntype{L}[1]{>{\raggedright\let\newline\\\arraybackslash\hspace{0pt}}m{#1}}
\newcolumntype{C}[1]{>{\centering\let\newline\\\arraybackslash\hspace{0pt}}m{#1}}
\newcolumntype{R}[1]{>{\raggedleft\let\newline\\\arraybackslash\hspace{0pt}}m{#1}}

\usefonttheme{professionalfonts}
\setbeamertemplate{theorems}[numbered]
\setbeamertemplate{bibliography item}{\insertbiblabel}
% \setbeamercovered{transparent}


\theoremstyle{definition}
% \numberwithin{subsection}{section}
\newtheorem{definisi}{Definisi}
\newtheorem{teorema}{TEOREMA}
\newtheorem{contoh}{Contoh}
\newtheorem{latihan}{Latihan}
\newcommand{\R}{\mathbb{R}}
\newcommand{\N}{\mathbb{N}}
\newcommand{\Z}{\mathbb{Z}}
\newcommand{\C}{\mathbb{C}}


\AtBeginEnvironment{contoh}{%
\setbeamercolor{block title}{use=example text,fg=white,bg=example text.fg!75!black}
\setbeamercolor{block body}{parent=normal text,use=block title example,bg=block title example.bg!10!bg}
}
\AtBeginEnvironment{definisi}{
\setbeamercolor{block title}{fg=white,bg=HIMAtua}
\setbeamercolor{block body}{parent=normal text,bg=HIMAtua!30!white}
}
\AtBeginEnvironment{latihan}{%
  \setbeamercolor{block title}{fg=white,bg=yellow!50!orange} % Set title background to pastel green and text to white
  \setbeamercolor{block body}{parent=normal text,bg=yellow!30!white} % Set body background to a lighter pastel green
}

\date{Sabtu, 8 Februari 2025}
\title[Aritmatika Modular]{Aritmatika Modular}
\author[Tew \& Haf]{Teosofi Hidayah Agung\\Hafidz Mulia}

\begin{document}
\begin{frame}
    \titlepage
\end{frame}

\begin{frame}{Daftar Isi}
    \tableofcontents
\end{frame}

\section{Modulo}

\begin{frame}
    \frametitle{\insertsection}
    \begin{definisi}
        Diberikan bilangan bulat $a$ dan $b$ serta bilangan bulat positif $m$. Kita katakan bahwa $a$ kongruen dengan $b$ modulo $m$ jika $m$ membagi $a-b$. Notasi untuk menyatakan hal ini adalah
        \[a\equiv b\pmod{m}\]
    \end{definisi}
    \onslide<2->{
        Definisi diatas menunjukkan bahwa jika kita menetapkan suatu nilai $a$ maka ada tak terhingga kemungkinan nilai $b$ agar berlaku hal diatas.
    }
\end{frame}

\begin{frame}
    \frametitle{\insertsection}
    \begin{contoh}
        Misalkan $a=5$ dan $m=3$. Maka kita dapat menyatakan bahwa
        \begin{align*}
            5&\equiv 2\pmod{3}\equiv 8\pmod{3}\equiv -1\pmod{3}\equiv \dots
        \end{align*}
        Karena
        \begin{align*}
            5-2&=3\\
            5-8&=-3\\
            5-(-1)&=6\\
            \vdots
        \end{align*}
        yang berarti $3$ membagi selisih antara $a-b$ dengan $b=...,-3,3,6,...$.
    \end{contoh}
\end{frame}

\begin{frame}
    \frametitle{\insertsection}
    Sifat dari kongruensi modulo adalah sebagai berikut:
    \begin{enumerate}
        \item Jika $a\equiv b\pmod{m}$ dan $b\equiv c\pmod{m}$ maka $a\equiv c\pmod{m}$.
        \item Jika $a\equiv b\pmod{m}$ dan $c\equiv d\pmod{m}$ maka $a\pm c\equiv b\pm d\pmod{m}$
        \item Jika $a\equiv b\pmod{m}$ dan $c\equiv d\pmod{m}$ maka $a\cdot c\equiv b\cdot d\pmod{m}$.
        \item Jika $a\equiv b\pmod{m}$ maka $a^n\equiv b^n\pmod{m}$ untuk setiap bilangan bulat positif $n$.
    \end{enumerate}
\end{frame}


\section{Pembagian Bilangan Bulat}
\begin{frame}
    \frametitle{\insertsection}
    \begin{definisi}
        untuk setiap pasangan bilangan bulat $a$ dan $b$ ($b>0$), ada dua bilangan bulat tunggal $q$ dan $r$ sehingga
        \[a = bq + r,\,\, 0 \leq r < b\]
        dengan $q$ disebut sebagai hasil bagi dan $r$ disebut sebagai sisa pembagian.
    \end{definisi}
    Dalam pemograman, sisa pembagian biasanya dilambangkan dengan operator modulo (\%). Misal \texttt{int a} dan \texttt{int b}, maka \texttt{a\%b=r} berarti sisa pembagian \texttt{a} dengan \texttt{b} adalah \texttt{r}.
\end{frame}

\begin{frame}
    \begin{latihan}
        Berapakah sisa jika \( 7^{2019} \) dibagi 8?
    \end{latihan}
    \onslide<2->{\textbf{Jawab:}
        \[7^{2019}  \mod 8 \equiv (-1)^{2019} \mod 8\equiv (-1) \mod 8= 7\]
        Jadi, \( 7^{2019} \) jika dibagi 8 maka akan bersisa 7.}
\end{frame}

\begin{frame}
    \begin{latihan}
        Berapakah sisa \( (54^{54}+55^{55}) \) jika dibagi 7?
    \end{latihan}
    \onslide<2->{\textbf{Jawab:}
    Jawab:
    \begin{align*}
        (54^{54}+55^{55}) \mod 7
        &= (8 \cdot 7 -2)^{54} \mod 7 + (8 \cdot 7 -1)^{55} \mod 7\\
        &= (-2)^{54} \mod 7 + (-1)^{55} \mod 7\\
        &= (-8)^{18} \mod 7 + (-1) \mod 7\\
        &= (-1)^{18} \mod 7 + 6 \mod 7\\
        &= 1 \mod 7 + 6 \mod 7\\
        &= (1+6) \mod 7=0
    \end{align*}
    Jadi, sisa pembagian \( 54^{54}+55^{55} \) dengan 7 adalah 0.
    }
\end{frame}



\end{document}
