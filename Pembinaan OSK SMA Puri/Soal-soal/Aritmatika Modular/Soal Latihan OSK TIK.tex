\documentclass[a4paper]{article}
\usepackage{amsmath,amsfonts,amssymb,amsthm}
\title{Lathan OSK TIK}
\author{Aritmatika Modular}
\date{8 Februari 2025}

\begin{document}
  \maketitle
  \begin{enumerate}
    \item\textbf{(OSK 2022)} Jika \( n \) adalah bilangan bulat positif yang memenuhi persamaan
    $
    7^{n^4 - 3n^2 - 4} = 11^{n^4 - 3n^2 - 4}
    $
    berapakah digit terakhir dari nilai \( n^{2022} \)?
    
    \begin{enumerate}
        \item[A.] 2
        \item[B.] 4
        \item[C.] 8
        \item[D.] 0
        \item[E.] 6
    \end{enumerate}

    \item\textbf{(OSK 2022)} Berapakah nilai digit terakhir dari $76257^5 \times 12617^7$?
    \begin{itemize}
        \item[A.] 1
        \item[B.] 3
        \item[C.] 5
        \item[D.] 9
        \item[E.] 7
    \end{itemize}

    \item\textbf{(OSK 2023)} Sebuah situs undian gratis berniat mengundi sebuah string berisi huruf dengan panjang 3. 
    Mereka membuka slot tak terbatas untuk menebaknya. Tebakan benar akan mendapat hadiah 10 miliar. 
    Pak DengkleK mengajak 17.576 bebeknya untuk ikut menebak dari AAA hingga ZZZ urut secara leksikografis 
    (bebek ke-1 menebak AAA, bebek ke-2 menebak AAB, bebek ke-17.576 menebak ZZZ) agar dipastikan memenangkan hadiah. 
    Maka bebek yang ke-1532 akan menebak string...
    \begin{itemize}
        \item[A.] CGX
        \item[B.] BFX
        \item[C.] BGX
        \item[D.] CGY
        \item[E.] AFX
    \end{itemize}
    
    \item \textbf{(OSK 2023)} Sebuah situs undian gratis berniat mengundi sebuah string berisi huruf dengan panjang 3. 
    Mereka membuka slot tak terbatas untuk menebaknya. Tebakan benar akan mendapat hadiah 10 miliar. 
    Pak Dengklek mengajak 17.576 bebeknya untuk ikut menebak dari AAA hingga ZZZ urut secara leksikografis 
    (bebek ke-1 menebak AAA, bebek ke-2 menebak AAB, bebek ke-17.576 menebak ZZZ) agar dipastikan memenangkan hadiah. 
    Jika string yang keluar adalah "OSN" maka bebek keberapa yang berhasil menebak dengan benar?
    \begin{itemize}
        \item[A.] 3528
        \item[B.] 3990
        \item[C.] 9945
        \item[D.] 9946
        \item[E.] 10648
    \end{itemize}

    \item \textbf{(OSK 2024)} Terdapat 8 buah lampu yang berjejer dari kiri ke kanan dan dilabeli dengan huruf A, B, C, D, E, F, G, dan H. Pada awalnya, setiap lampu dalam kondisi mati.

    Terdapat 10 bebek Pak Dengklek yang ingin bermain dengan lampu-lampu tersebut. Diketahui, bahwa masing-masing dari mereka memiliki lampu favorit masing-masing yang ditandai dengan centang pada tabel di bawah.
    
    \begin{table}[h]
        \centering
        \begin{tabular}{|c|c|c|c|c|c|c|c|c|}
            \hline
            \textbf{Bebek} & \textbf{A} & \textbf{B} & \textbf{C} & \textbf{D} & \textbf{E} & \textbf{F} & \textbf{G} & \textbf{H} \\
            \hline
            1  & \checkmark &   & \checkmark & \checkmark  &   &  \checkmark & &  \checkmark \\\hline
            2  &   & \checkmark & \checkmark  &  & \checkmark  &  & \checkmark  &  \checkmark \\\hline
            3  & \checkmark &   &   & \checkmark  &  &  \checkmark &  &  \checkmark \\\hline
            4  &   & \checkmark & \checkmark &   &   &  \checkmark &  \checkmark &   \\\hline
            5  & \checkmark & & \checkmark  &   & \checkmark  & \checkmark &   &   \\\hline
            6  &   &   & \checkmark & \checkmark &   &  \checkmark & \checkmark &   \checkmark\\\hline
            7  & \checkmark &  \checkmark &   &  \checkmark &  \checkmark &  & \checkmark  & \checkmark \\\hline
            8  &  \checkmark & \checkmark &  &   & \checkmark &   & \checkmark  &   \\\hline
            9  &  &  \checkmark &  & \checkmark  & \checkmark  &   &   &  \\\hline
            10 &  \checkmark &   & \checkmark  & &   & \checkmark & \checkmark &   \\
            \hline
        \end{tabular}
    \end{table}
    
    Pak Dengklek memilih 9 dari 10 bebeknya. Untuk setiap bebek yang terpilih, satu per satu, ia akan mengganti seluruh status lampu-lampu favoritnya: jika sebelumnya mati, maka akan berubah menjadi menyala; jika sebelumnya menyala, maka akan berubah menjadi mati.
    
    Jika ternyata kondisi akhirnya adalah hanya lampu A dan G yang menyala, bebek nomor berapakah yang \textbf{tidak dipilih} oleh Pak Dengklek?
    
    \textbf{\underline{Tuliskan jawaban dalam bentuk ANGKA}.}
  \end{enumerate}
\end{document}