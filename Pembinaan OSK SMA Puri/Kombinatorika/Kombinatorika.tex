\documentclass[aspectratio=169]{beamer}
\usepackage{colortbl,tabularx,mathrsfs,calligra}
\usepackage{amsmath,amsfonts,amssymb,amsthm}
\usepackage{ragged2e}
\usepackage[bahasa]{babel}
\usepackage{tikz}
\usepackage{caption}
\usepackage{wrapfig}
\usepackage{multirow}
\usepackage{multicol}
\usepackage{array}
\usepackage{ulem}
\usepackage{pgfplots, tkz-euclide,calc}
\pgfplotsset{compat=1.18}
\usepackage{listings}

\graphicspath{{C:/Users/teoso/OneDrive/Documents/Tugas Kuliah/Template Math Depart/}{./foto/}}

\definecolor{HIMAmuda}{HTML}{01D1FD}
\definecolor{HIMAtua}{HTML}{02016A}
\definecolor{HIMAabu}{HTML}{CBCBCC}
\definecolor{PastelGreen}{HTML}{77DD77}

\usetheme{Madrid}

\setbeamercolor{palette primary}{bg=HIMAtua,fg=white}
\setbeamercolor{palette secondary}{bg=HIMAmuda,fg=black}
\setbeamercolor{palette tertiary}{bg=HIMAabu,fg=black}
\setbeamercolor{palette quaternary}{bg=HIMAmuda,fg=white}
\setbeamercolor{structure}{fg=HIMAmuda} % itemize, enumerate, etc
\setbeamercolor{section in toc}{fg=HIMAtua} % TOC sections
\setbeamercolor{bibliography item}{parent=palette secondary}
\setbeamercolor*{bibliography entry author}{parent=section in toc}

\usetikzlibrary{shapes.geometric, arrows}

\tikzstyle{startstop} = [ellipse, minimum width=1cm, minimum height=1cm,text centered, draw=black, fill=red!30]
\tikzstyle{process} = [rectangle, minimum width=2cm, minimum height=1cm, text centered, draw=black, fill=blue!30]
\tikzstyle{decision} = [diamond, minimum width=1cm, minimum height=1cm, text centered, draw=black, fill=blue!50]
\tikzstyle{arrow} = [thick,->,>=stealth]

\newcolumntype{L}[1]{>{\raggedright\let\newline\\\arraybackslash\hspace{0pt}}m{#1}}
\newcolumntype{C}[1]{>{\centering\let\newline\\\arraybackslash\hspace{0pt}}m{#1}}
\newcolumntype{R}[1]{>{\raggedleft\let\newline\\\arraybackslash\hspace{0pt}}m{#1}}

\usefonttheme{professionalfonts}
\setbeamertemplate{theorems}[numbered]
\setbeamertemplate{bibliography item}{\insertbiblabel}
% \setbeamercovered{transparent}


\theoremstyle{definition}
% \numberwithin{subsection}{section}
\newtheorem{definisi}{Definisi}
\newtheorem{teorema}{Teorema}
\newtheorem{contoh}{Contoh}
\newtheorem{latihan}{Latihan}
\newcommand{\R}{\mathbb{R}}
\newcommand{\N}{\mathbb{N}}
\newcommand{\Z}{\mathbb{Z}}
\newcommand{\C}{\mathbb{C}}


\AtBeginEnvironment{contoh}{%
\setbeamercolor{block title}{use=example text,fg=white,bg=example text.fg!75!black}
\setbeamercolor{block body}{parent=normal text,use=block title example,bg=block title example.bg!10!bg}
\setbeamercolor{item}{fg=example text.fg}
}
\AtBeginEnvironment{definisi}{
\setbeamercolor{block title}{fg=white,bg=HIMAtua}
\setbeamercolor{block body}{parent=normal text,bg=HIMAtua!30!white}
\setbeamercolor{item}{fg=HIMAtua}
}
\AtBeginEnvironment{latihan}{%
  \setbeamercolor{block title}{fg=white,bg=yellow!50!orange} 
  \setbeamercolor{block body}{parent=normal text,bg=yellow!30!white} 
  \setbeamercolor{item}{fg=yellow!50!orange}
}
\AtBeginEnvironment{teorema}{
  \setbeamercolor{block title}{bg=darkgray,fg=white}
  \setbeamercolor{block body}{parent=pallette tertiary,bg=HIMAabu!30!white}
  \setbeamercolor{item}{fg=white}
}

\date{Sabtu, 8 Maret 2025}
\title[Kombinatorika]{Permutasi \& Kombinasi}
\author[Tew \& Haf]{Teosofi Hidayah Agung\\Hafidz Mulia}

\begin{document}
\begin{frame}
    \titlepage
\end{frame}

\begin{frame}{Apa itu Kombinatorika?}
  \textbf{Kombinatorika} adalah cabang matematika yang mempelajari sifat serta metode \textcolor{red}{perhitungan (\textit{counting})} berbagai struktur \textcolor{red}{berhingga (\textit{finite})}. Cabang ini dapat dijelaskan melalui beberapa jenis permasalahan yang biasanya dikaji, yaitu:
  \begin{itemize}
    \item \textcolor{red}{Menghitung jumlah kemungkinan} struktur atau susunan dalam suatu sistem hingga.
    \item Menentukan apakah terdapat struktur yang \textcolor{red}{memenuhi syarat} tertentu.
    \item Mengonstruksi struktur-struktur tersebut dengan \textcolor{red}{berbagai cara}.
    \item \textcolor{red}{Mengoptimalkan struktur} atau solusi agar memenuhi kriteria tertentu.
  \end{itemize}
\end{frame}

\begin{frame}{Daftar Isi}
    \tableofcontents
\end{frame}

\section{Aturan Penjumlahan}
\begin{frame}
\frametitle{\insertsection}
\begin{definisi}
  Jika ada sebanyak $m$ pilihan pada kejadian pertama dan ada sebanyak $n$ pilihan pada kejadian kedua, dan kedua kejadian itu \textbf{tidak dapat dilakukan dalam waktu yang sama}, maka ada 
\begin{align}
    m+n
\end{align} 
cara untuk memilih satu dari kejadian tersebut. Biasanya pada soal terdapat kata \textbf{``atau''}.
\end{definisi}
Untuk kasus umum, jika ada sebanyak $n$ kejadian yang tidak dapat dilakukan dalam waktu yang sama, maka ada
\begin{align}
    n_1+n_2+...+n_k
\end{align}
kemungkinan.
\end{frame}

\begin{frame}
  \frametitle{\insertsection}
  \begin{contoh}
    Berapa banyak cara untuk memilih untuk membaca antara 3 buku matematika, 4 buku fisika, dan 2 buku kimia? 
  \end{contoh}
  \textbf{Jawab:} $3+4+2=9$.
  \begin{contoh}
    Banyak cara membeli sebuah piring dari 6 piring plastik atau 4 piring kaca
  \end{contoh}
  \textbf{Jawab:} $6+4+2=10$.
\end{frame}

\section{Aturan Perkalian}
\begin{frame}
\frametitle{\insertsection}
\begin{definisi}
  Jika ada sebanyak $m$ pilihan pada kejadian pertama dan ada sebanyak $n$ pilihan pada kejadian kedua, dan kedua kejadian itu \textbf{dilakukan dalam waktu yang sama}, maka ada
  \begin{align}
      m \times n
  \end{align}
  cara untuk memilih satu dari kejadian tersebut. Biasanya pada soal terdapat kata \textbf{``dan''}.
\end{definisi}
Untuk kasus umum, jika ada sebanyak $n$ kejadian yang dilakukan dalam waktu yang sama, maka ada
\begin{align}
    n_1 \times n_2 \times ... \times n_k
\end{align}
kemungkinan.
\end{frame}

\begin{frame}
  \frametitle{\insertsection}
  \begin{contoh}
    Misalkan terdapat $2$ buah baju dan $3$ buah celana. Berapa banyak seseorang dapat memilih baju dan celana yang akan Ia pakai?
  \end{contoh}
  \textbf{Jawab:} $2\times 3=6$.
  \begin{contoh}
    Terdapat empat jalan yang menghubungkan kota P dan kota Q, tiga jalan yang menghubungkan kota Q dan kota R serta tiga jalan dari kota R ke kota S. Tentukanlah banyaknya rute perjalanan seseorang dari koa P ke kota S !
  \end{contoh}
  \textbf{Jawab:} $4\times 3\times 3= 36$.
\end{frame}

\subsection{\textit{Filling Slot}}
\begin{frame}
\frametitle{\insertsection}
\framesubtitle{\insertsubsection}
Aturan perkalian sering kali disebut dengan aturan pengisian tempat (\textit{\textcolor{red}{filling slot}}). Dibawah ini merupakan \textbf{POV} lain dari aturan perkalian.
\begin{definisi}
  Misalkan ada $n$ tempat yang tersedia, dengan tempat ke-$1$ memiliki cara sebanyak $k_1$, tempat ke-$2$ memiliki cara sebanyak $k_2$, dan seterusnya sampai tempat ke-$n$ memiliki cara sebanyak $k_n$. Dengan demikian, banyaknya cara mengisi tempat adalah 
  \begin{align}
      \uline{k_1} \ \uline{k_2} \ \uline{k_3} \ ... \ \uline{k_n} = k_1 \times k_2 \times k_3 \times ... \times k_n.
  \end{align}
\end{definisi}

\end{frame}

\begin{frame}
\frametitle{\insertsection}
\framesubtitle{\insertsubsection}
  \begin{contoh}
    Banyaknya cara membuat string 8 karakter yang terdiri dari huruf `W' dan `K'. 
  \end{contoh}
  \onslide<2->{\textbf{Jawab:} $\uline{2} \ \uline{2} \ \uline{2} \  \uline{2} \ \uline{2} \ \uline{2} \  \uline{2} \ \uline{2}=2^8$.}
  \begin{contoh}
    Banyaknya menyusun angka $1,2,5,7,8,0$ menjadi sebuah bilangan ratusan. 
  \end{contoh}
  \onslide<2->{\textbf{Jawab:} $\ \uline{5} \  \uline{5} \ \uline{4}=100$.}
  \begin{contoh}
    Berapa string yang dapat dibuat dari karakter `I', `T', `S'? 
  \end{contoh}
  \onslide<2->{\textbf{Jawab:} $\ \uline{3} \  \uline{2} \ \uline{1}=6$.}
\end{frame}

\subsection{Faktorial}
\begin{frame}
\frametitle{\insertsection}
\framesubtitle{\insertsubsection}
Sebelum masuk pada permutasi dan kombinasi, akan diperkenalkan terlebih dahulu definisi dan notasi faktorial. 
\begin{definisi}
  Notasi $n!$ dibaca \textbf{$n$ faktorial} yang didefinisikan sebagai
    \begin{align}
        n!=n \times (n-1) \times (n-2) \times ... \times 3 \times 2 \times 1
    \end{align}
    untuk setiap $n$ bilangan asli.
\end{definisi}

Contoh :
    \begin{align*}
        5! &=5 \times 4 \times 3 \times 2 \times 1 = 120 \notag \\
        8! &= 8 \times 7 \times 6 \times 5 \times 4 \times 3 \times 2 \times 1 = 40320
    \end{align*}
\end{frame}

\section{Permutasi}
\begin{frame}
\frametitle{\insertsection}
\begin{definisi}
    Permutasi (\textbf{susunan}) $r$ objek yang diambil dari $n$ objek berbeda adalah $P_r^n$ yang didefinisikan dengan
    \begin{align}
        P_r^n = \frac{n!}{(n-r)!}
    \end{align}
    dibaca "$n$ permutasi $r$".
\end{definisi}
Permutasi biasanya jarang digunakan karena terkesan tidak fleksibel seperti \textit{filling slot}. Namun permutasi menjadi cikal bakal dari \textbf{kombinasi} yang akan kita bahas selanjutnya.
\end{frame}

\begin{frame}
  \frametitle{\insertsection}
  \begin{contoh}
    Berapa banyak cara menyusun 3 huruf berbeda dari 4 huruf K, L, M, N ?
  \end{contoh}
  \textbf{Penyelesaian : }\\
  Jika menggunakan \textit{filling slot} yang sebelumnya kita ketahui, maka banyak caranya adalah 
  \begin{align}
       \uline{4} \ \uline{3} \  \uline{2} = 4 \times 3 \times 2 = 24 \notag
  \end{align}
  Atau, karena kita mengambil 3 dari 4 objek berbeda, akibatnya kita dapat menggunakan permutasi dari unsur-unsur yang berbeda, dengan $r=3$ dan $n=4$.
  \begin{align}
      P_3^4 = \frac{4!}{(4-3)!}=\frac{4!}{1!}=\frac{4 \times 3 \times 2 \times 1}{1}=24. \notag
  \end{align}
\end{frame}

\subsection{Unsur yang Sama}
\begin{frame}
  \frametitle{\insertsection}
  \framesubtitle{\insertsubsection}
  \begin{teorema}
    Banyak permutasi $n$ unsur yang memuat $k_1$ unsur yang sama, $k_2$ unsur yang sama, dan seterusnya sampai dengan $k_i$ unsur yang sama, dengan $k_1+k_2+k_3+...+k_i=n$ ditentukan dengan rumus
\begin{align}
    P^n_{k_1,k_2,\dots,k_i}=\frac{n!}{k_1!k_2!...k_i!}
\end{align}
  \end{teorema}
\end{frame}

\begin{frame}
  \frametitle{\insertsection}
  \framesubtitle{\insertsubsection}
  \begin{contoh}
    Tentukan banyak susunan yang dapat dibentuk dari huruf-huruf berikut:
    \begin{enumerate}
      \item `M',`O',`J',`O',`K',`E',`R',`T',`O'
      \item `M',`A',`T',`E',`M',`A',`T',`I',`K',`A'
      \item `P',`U', `L', `L', `U', `P'
    \end{enumerate}
  \end{contoh}
  \onslide<2->{\setbeamercolor{item}{fg=example text.fg}
    \textbf{Jawab:}
  \begin{enumerate}
    \item $\displaystyle P^9_{3}= \frac{9!}{3!}$
    \item $\displaystyle P^{10}_{2,3,2}= \frac{10!}{2!3!2!}$
    \item $\displaystyle P^6_{2,2,2}= \frac{6!}{2!2!2!}$
  \end{enumerate}}
\end{frame}

% \begin{frame}{Permutasi}
% \hypertarget{siklis}{}
% \justifying
% \textbf{Permutasi Siklis}\\
% Bagaimana jika ada $n$ orang duduk melingkari meja bundar? Ada berapa banyak cara menyusunnya?\\
% Banyaknya permutasi siklis dari $n$ unsur tersebut dirumuskan dengan
% \begin{align}
%     P(\text{siklis}) = (n-1)!
% \end{align}
% \textbf{Contoh : }\\
% Jika terdapat tiga orang yang duduk pada tiga kursi yang membentuk suatu lingkaran, maka ada berapa banyak susunan yang mungkin terjadi?\\
% \textbf{Penyelesaian : }\\
% Permutasi siklis dengan $n=3$ didapat\\
% \begin{align}
%     P(\text{siklis})=(3-1)!=2!=2. \notag
% \end{align}
% \end{frame}

\section{Kombinasi}
\begin{frame}
\frametitle{\insertsection}
\begin{definisi}
  Suatu kombinasi $r$ unsur yang diambil dari $n$ unsur berbeda yang tersedia adalah suatu pilihan dari r unsur tadi\textbf{ tanpa memperhatikan urutannya}. Banyaknya kombinasi $r$ unsur yang diambil dari $n$ unsur yang tersedia dengan $r \leq n$ dirumuskan dengan
  \begin{align}
      C_r^n=\frac{n!}{(n-r)!r!}
  \end{align}
  atau dapat juga ditulis sebagai $\displaystyle \binom{n}{r}$ dan biasanya dibaca sebagai "$n$ kombinasi $r$" atau "$n$ dipilih $r$".
\end{definisi}
Pada dasarnya ini adalah permutasi dengan menganggap $r$ objek sejenis, sehingga aturan penyusunan tidak berlaku untuk objek yang sama.
\end{frame}

\begin{frame}
  \frametitle{\insertsection}
  \begin{contoh}
    Berapa banyak cara memilih 2 perwakilan lomba dari 5 siswa yang ada?
  \end{contoh}
  \onslide<2->{\textbf{Jawaban:} $\displaystyle C_2^5=\frac{5!}{(5-2)!2!}=\frac{5!}{3!2!}=10$.}
  \begin{contoh}
    Berapa banyak cara memilih 3 buah kartu dari 52 kartu poker?
  \end{contoh}
  \onslide<2->{\textbf{Jawaban:} $\displaystyle C_3^{52}=\frac{52!}{(52-3)!3!}=\frac{52!}{49!3!}=22100$.}
  \begin{contoh}
    Tim sepak bola memiliki 14 pemain dan pelatih ingin memilih 11 pemain untuk diturunkan.
  \end{contoh}
  \onslide<2->{\textbf{Jawaban:} $\displaystyle C_{11}^{14}=\frac{14!}{(14-11)!11!}=\frac{14!}{3!11!}=364$.}
\end{frame}

\begin{frame}
  \frametitle{\insertsection}
  \begin{teorema}
    Untuk setiap bilangan bulat non-negatif $n$ dan $r$ dengan $r \leq n$, berlaku
    \begin{align}
        C_r^n=C_{n-r}^n
    \end{align}
  \end{teorema}
  Bukti kombinatorial dari teorema ini adalah dengan menganggap jika kita memilih $r$ objek dari $n$ objek yang tersedia, maka banyaknya cara memilih $r$ objek adalah sama dengan banyaknya cara memilih $n-r$ objek yang tidak dipilih.
\end{frame}

\end{document}